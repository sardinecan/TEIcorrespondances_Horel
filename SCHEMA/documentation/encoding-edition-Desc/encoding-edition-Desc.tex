\documentclass[12pt,a4paper,oneside]{book} % ou article, memoir, report, etc.

%usepackage permet d'utiliser un module complémentaire.

% Marges, retraits et espacement
\usepackage[margin=2.5cm]{geometry}
%espacement des lignes
\usepackage{setspace}
\onehalfspacing
\setlength\parindent{1cm}

%Le package Babel permet de gérer différentes normes linguistiques et typographiques
\usepackage[english,italian,french]{babel}


%Nouvelle syntaxe ici: une commande avec une option entre []
\usepackage[utf8]{inputenc}

%Gérer l'encodage des caractères en sortie
\usepackage[T1]{fontenc}

%Changer la fonte de caractère
\usepackage{lmodern}
%\usepackage{charter}
%Pour les maniaques du Times \usepackage{txfonts}
%Palatino \usepackage{mathpazo}

%Métadonnées du document
%Auteur
\author{}
\title{}
%La commande date est optionnelle
%\date{2 janvier 1950}
\date{Version du \today}


%Écrire les mots inconnus du dictionnaire pour la césure/l'hyphénation
%\hyphenation{Ajou-t-ons}


%Autres modules complémentaires
\usepackage{lettrine}
\usepackage[pdftex]{graphicx}

%pour le mode paysage je dois utiliser deux packages
%\usepackage{lscape}%pour le mode paysage
%\usepackage{pdflscape}%pour l'indiquer dans les métadonnées du pdf.

%pour le dessin
\usepackage{tikz}%pour le dessin
\usepackage{tikz-qtree}%pour les arbres 

\usepackage{caption}%pour ne pas numéroter les figures (on ajoute une étoile après caption)
\usepackage{listings}
%pour créer un Index
\usepackage{makeidx}
%pour activer la création de l'index.
\makeindex 

%Appeller le package pour les bibliographies
\usepackage[backend=biber, sorting=nyt, style=inha]{biblatex}
%appel de la ressource bibliographique.
\addbibresource{memoire.bib}
%pour les guillemets
\usepackage[babel]{csquotes}

\begin{document}

\part*{Documentation}
\chapter*{encodingDesc}
\section*{segmentation}

Description des principes selon lesquels le texte a été segmenté, par exemple en phrases, en intonèmes (unités tonales), en strates graphématiques (niveaux superposés de signes graphiques), etc.
\bigskip 


Toute édition doit être contenue dans un élément <text> précisé par les attributs @xml:id et @n afin d'identification. L'élément <body> et son sous-élément <div type="letter">\footnote{Ils peuvent aussi être complétés par un attribut @xml:id.} contiendront la transcription de la lettre. Un élément <back> peut être intégré pour la gestion des notes et des index.

Les éventuelles enveloppes pourront être encodé dans le <body>, dans un sous-élément <div type="enveloppe"> avec un attribut @facs

\subsection*{Corps de la correspondance}

Il convient de rendre compte, lors de l'encodage et dans la mesure du possible, de la structure diplomatique des dépêches.\footcite{desenclos_edition_????}
\bigskip 

Les formules d'ouverture, seront placées dans un élément <opener>. Les mentions de dates et de lieux trouveront leur place au sein d'un sous-élément <dateLine> alors que l'apostrophe sera encodée à l'aide de la balise <salute>.

Le corps de la lettre sera encodé dans un élément <p>. Les changements de ligne, de colonne, de page ou encore de support seront matérialisés par l'emploi respectif des balises <lb/>, <cb/>, <pb/>, <gb/>; ils pourront être complétés par l'utilisation des attributs @n et @facs afin de leur associer une représentation fac-similaire par exemple. 

Les formules de fermeture de la correspondance seront placées dans un élément <closer> pouvant contenir les formules de politesse (<salute>), les mentions de date (<dateline>), la signature (<signed>) ou encore une adresse (<address> avec les sous-éléments <addrLine>). Le <closer> peut être répété et typé afin de prendre en compte les mentions autographes par exemple.\footcite{desenclos_edition_????}%ne faut-il pas mettre les dernières addresse dans un note ? avec un type address ? cela peut être interessant mais dans certain peu poser problème.

Le postscriptum (<postscript>) peut être placé avant ou après le <closer> et il peut prendre la même forme qu'une correspondance (utilisation de formules d'ouverture, de fermeture et de signature.).

\lstset{language=XML}
\begin{lstlisting}
<text>
  <body>
    <div type="letter">
      <opener>
        <dateline>Paris, le 28 mars 1916</dateline>
        <salute>Madame,</salute>
      </opener>
      <p>
        Ici le corps de la lettre
        <lb/>
        <pb/>....
      </p>
      <closer>
        <salute/>
        <signed/>
      </closer>
      <postscript>
        <opener/>
        <p/>
        <closer/>
      </postscript>
    </div>
  </body>
  <back>
    <div type="note">
      <note/>
    </div>
  </back>
</text>
\end{lstlisting}
\pagebreak

\section*{Ponctuation}
 
Les éléments de ponctuation seront encodés avec la balise <pc>. À toutes fins de normalisation, la ponctuation ajoutée sera associée à l'attribut @type="supplied" ; la ponctuation superflue sera typée @type="surplus". Enfin en cas de modification d'une ponctuation existante, nous utiliserons un élément <choice> et ses sous-éléments <orig> et <reg> afin d'identifier la forme originelle de la forme régularisé.
\bigskip 

La modernisation de la ponctuation devra s'accompagner du rétablissements des majuscules et des minuscules à l'aide de la balise <hi> et de son attribut @rend="capitalize /minimize".\footnote{Voir section suivante}

\section*{Correction}

Concernant les corrections orthographiques, elles seront encodées avec une élément <choice> et les couples <reg>/<orig> et <sic>/<corr>. Une distinction sera faite entre, les fautes systématiques (<reg>/<orig>) et les fautes occasionnelles éparses (<sic> /<corr>)\footcite{nougaret_ledition_2015}.

\section*{Normalisation}
 
Afin de pouvoir offrir une lecture modernisée du texte, certains termes doivent être normalisés :
\begin{itemize}
\item Les nombres devront aussi être encodés en toutes lettres à quelques exceptions près: 
	\begin{itemize}
	\item En chiffres arabes : les dates, énumérations, numéros d'ordre (adresse, matricule\dots), calibres, degrés, échelles pourcentages\dots
	\item En chiffre romains : siècles, numéros dynastiques, régions politiques, régions militaires, division d'un ouvrage\dots.
	\item Utilisation des balises <choice> et <reg>/<orig>
	\end{itemize}
\item Les majuscules et minuscules doivent être rétablies par l'utilisation de la balise <hi> et de sont attributs @rend="capitalize/minimize" une attention particulière portera sur :
	\begin{itemize}
	\item La majuscule en début de phrase; 
	\item La majuscule pour l'initiale d'un nom propre;
	\item Les minuscules pour les noms communs.
	\end{itemize}
\item Il faut fournir, dans la mesure du possible, une version développée des abréviations utilisées (à l'exception des plus courantes) par l'emploi de l'élément <choice> et des sous-éléments <abbr> et <expan>\footnote{Il n'est pas besoin d'utiliser l'élément <ex> pour encoder les lettre ajouter pour le développement de l'abréviation.}.
\item Les coupures et espacement des mots doivent aussi être normalisés. 
	\begin{itemize}
	\item L'élément <space/> peut être utilisé pour sémantiser un espace 
	\item Pour la séparation des mots unis dans la source, l'élément <w> avec un attribut @rend="aggl" sera utilisé. 
	
	Exemple : enroute > <w rend="aggl">en</w><w>route</w>
	\item Pour la séparation des mots donnant lieu à une élision, la même balise sera utilisée mais avec l'attribut @rend="elision". 
	
	Exemple : lun > <w rend="elision">l</w><w>un</w>
	\item Enfin, pour la réunion de deux mots séparés dans la source, une simple élément <w> sera employé. 
	
	Exemple : criti que > <w>criti que</w>   
	\end{itemize}
\end{itemize}

\section*{hyphénation}

les cas d'hyphénation seront matérialisés par l'attribut @rend="hyphen" appliqué à l'élément linebreak concerné (<lb/>).

\subsection*{généralité}
 
\subsubsection*{Les Dates}
Les dates seront encodées avec l'élément Date, précisés par les attributs @when (format : AAAA-MM-JJ) et en cas d'incertitudes @cert et @nobefore / @notafter pour donner une échelle. 

\subsubsection*{Noms}

Les noms de personne et de lieu seront encodé avec les éléments <persName> et <placeName> pouvant être complété par l'attribut @ref si l'on dispose d'une entrée d'index.

pour les autres noms, l'élément <name> pourra être utilisé avec les attributs @type pour le préciser et @ref pour l'indexation.

\subsubsection*{Les ajouts - suppressions - oublis - manques - incertitudes}

Les ajouts de l'auteur seront encodé avec l'élément <add> complété par les attributs @place pour l'emplacement, @rend pour l'orientation du texte, @type pour en préciser la nature et éventuellement un @n pour l'ordre de lecture. les ajouts du transcripteur ou de l'éditeur seront encodé avec l'élément <supplied> et ses attributs @reason afin d'expliquer l'ajout et @cert pour la certitude du terme ajouté. Les ajouts d'une autre main seront matériélisé par l'élément <handshift/> aidé des attributs @medium, @rend et @resp lorsque c'est possible.
\bigskip 

Les parties de textes effacées, biffées, ou raturées seront encodées avec l'élément <del> et les attributs @rend, @hand @cause, @cert, @rest. Il est ainsi possible d'identifier la main qui a causé l'effacement et d'en préciser les raisons. Les suppressions liées à au transcripteur ou à l'éditeur seront encodées avec l'élément <surplus> et l'attribut @reason
\bigskip 

Les termes peu clairs, seront traités avec l'élément <unclear> précisé des attributs @reason, @agent, @hand et @cert.
\bigskip 

Enfin les manques seront matérialisés par la balise <gap/> avec l'attribut @reason="missing" 

\subsubsection*{Adresses et cachets - timbres}

Les adresses sont encodées avec l'élément <address> et le sous-élément <addrLine> en cas de répétition par exemple avec une correspondance tenant sur plusieurs carte la balise <address> sera complétée par l'attribut @n.

Les cachets et timbres seront encodé avec l'élément <stamp> et l'attribut @type pour les préciser. La balise <orgName> peut être utilisée pour préciser l'organisme à l'origine des cachets par exemple.

\subsubsection*{Autres}

Les éléments en exposant seront encodé avec l'élément <hi> et l'attribut @rend ="super"
\bigskip 

Les ajoutées dans le textes seront placées dans des éléments <note> du back comme nous l'avons vu plus haut et complétés par les attributs @xml:id et @n.
Afin de faire le lien dans le corps de la dépêche, un élément <ref> poitant avec sont attribut @target vers le note et le même @n que cette dernière seront ajoutés. 
\bigskip 

Les images, photographies ou dessins, seront placés dans un élément <figure> comportant une balise <graphic> avec un attribut @url pour cibler le facsimilé et une balise <figDesc> pour la description de l'image.


\end{document}

